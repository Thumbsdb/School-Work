\documentclass{article}

\usepackage{graphicx}

\usepackage{hyperref}
\hypersetup{
    colorlinks=true,
    linkcolor=blue,
    filecolor=magenta,      
    urlcolor=cyan,
    pdftitle={Overleaf Example},
    pdfpagemode=FullScreen,
}
\usepackage{titlesec}
\usepackage{geometry}
 \geometry{
 a4paper,
 total={170mm,257mm},
 left=20mm,
 top=20mm,
 }

\title{Unit 15 P2}
\author{Chris}
\date{}

\begin{document}

\maketitle
\tableofcontents
\break
 

\section{Benefits of prototyping}
Prototyping marks the creation of your product's main design and how it is going to work before initial development starts off. This vital step ensures the product operates as expected and meets the requirements.


The key advantage of prototyping is its capacity to reveal design flaws and manufacturing challenges. Identifying these issues early can help complications later down in the production, significantly reducing wasted resources and the risk of creating subpar products.


Moreover, engaging with users through prototype feedback illuminates user experiences, preferences, and potential issues. This interaction allows for well-informed enhancements, ensuring the product is finely tuned to user expectations. Prototyping not only streamlines the development process but also boosts the final product's usability and success, aligning closely with user demands and enhancing overall satisfaction.


Here is a list of potential benefits of prototyping:
\begin{enumerate}
    \item It's easier to discover possible design defects
    \item Selecting machinery and other production resources is simplified because prototypes can be very similar
    \item The expected durability of the product is known in advance thanks to the tests which the prototype is subjected to
    \item You can't go into production without feedback from the end customer. 
\end{enumerate}

\section{Types of prototyping}
The different types of prototyping are :
\begin{enumerate}
    \item Sketches and diagrams / story boarding
    \item 3D printing or rapid model
    \item Physical model
    \item Wire fame
    \item Role-play through virtual or augmented reality
    \item Feasibility
    \item Working model
    \item Video prototype
\end{enumerate}


\section{Testing concepts}
There are a number of concepts to grasp when testing in general and testing games are being researched. Probably the most important is the player experience which involves trying to understand if the game is enjoyable or not. This can be very subjective which means that while this is the most important concept, it is the hardest to measure and to normalise.
Next we need to understand code logic testing. We need to make sure that the game does not crash and will work on different platforms and and with different controllers.
There are also different methods of testing, such as A/B testing where different users are playing the same game but with, for example, slightly different UX. This allows us to compare and contrast different solutions and to measure how much players rate the changes.


\section{Measuring player engagement levels}
Measuring player engagement is critical to ensure they remain interested and invested in your game. This represents one of the most significant challenges in game development. If player retention has not been a priority, you're likely to face financial losses and a shrinking player community. It's essential to incorporate compelling features in your game that motivate players to revisit and enjoy your game multiple times. You may wonder about the factors that contribute to a game's allure and the methods to keep players engaged. To achieve this, it's necessary to explore innovative game concepts and practical tips for player retention.



You might question what elements make a game appealing and how to ensure players remain engaged. To tackle these questions, we have gathered a selection of intriguing game ideas and strategies for effective player retention.

\begin{enumerate}
    \item League tables
    \item Loot boxes
    \item Tournaments
    \item LAN parties
    \item Level ups
\end{enumerate}

\section{Necessary expertise level for successful participation and performance of a game}
The complexity of a project and the skills needed to complete it often vary based on the type of project being undertaken. For example, developing a 2D platformer game is generally considered more accessible due to the abundance of tools and assets available to assist in its creation. In contrast, producing a triple-A game poses a greater challenge due to the scarcity of ready-made assets.


Creating a prototype for a 2D platformer requires relatively basic skills, thanks to the wider variety of tools available for design and development. Some of the tools you can utilise for crafting a 2D platformer include Pixel Game Maker MV, RPG Maker MZ, 001 Game Creator, SRPG Studio, Solarus, and Wolf RPG Editor. This abundance of resources simplifies the process significantly, as highlighted by the variety of tools listed.


Developing a prototype for a Triple A game is significantly more challenging than for a 2D platformer, largely due to the scarcity or absence of generic tools for Triple A game creation. This scarcity forces that game designers and artists build the game's content and assets entirely from the ground up. However, this constraint offers designers considerable more creative freedom in the game development process compared to a 2D platformer. When it comes to selecting a game engine, options are typically restricted or costly. Developers often have to decide between using their proprietary engine or opting for widely-used platforms like Unreal Engine or Unity.


\section{Measuring achievement's level of difficulty}
Difficulty is one of the essential elements of all achievements. But what is difficulty? One might think that it involves the complexity of an activity. But it turns out that complexity is not a necessary for difficulty. Rather, we develop an account of difficulty according to which an action is difficult for an agent in that it takes a certain amount of effort on the part of that agent, where the effort is above a certain degree of intensity. How much effort is relevant for difficulty is relative to the particular activity. Although you might think that, say, a virtuoso can do something difficult with no effort, this turns out to be not quite right. Rather, difficulty is relative to a class. What the virtuoso does is difficult for most people, but not for the virtuoso. Here we present a comprehensive analysis of the nature of difficulty in terms of effort.

\begin{enumerate}
    \item Difficulty is a fundamental aspect of all achievements, yet its definition can be elusive. 
    \item Often, it's mistaken for the complexity of a task, but complexity isn't inherently what makes something difficult.
    \item Difficulty is more accurately defined by the amount of effort an agent must exert, specifically when this effort surpasses a certain intensity level. 
\end{enumerate}

\section{Clarity in user interface design}
The User Interface (UI) plays a vital role in the development of a successful game. It represents the primary means of interaction for players, significantly influencing their immersion and overall experience. Designing the UI for a game involves crafting the elements that users interact with directly. An intuitive and appealing UI enhances the gameplay, making it more engaging and accessible, whereas a UI that is not well thought out can lead to frustration and hinder the ease with which players navigate the game.


The top eight principles for a good UI design
\begin{enumerate}
    \item Clarity
    \item Consistency
    \item Feedback
    \item Responsiveness
    \item Accessibility
    \item Simplicity 
\end{enumerate} 

\section{Understanding gameplay and objectives clearly}

The integration of gaming into schools is progressively becoming a popular strategy for enhancing the learning experience, aiming to equip individuals with new skills in a more engaging and effective manner. The rationale behind this trend is anchored in the belief that interactive and immersive game-based learning environments can significantly improve motivation and facilitate the acquisition of complex concepts through practical application. Among the essential elements that contribute to the success of educational games, clarity stands out as a fundamental design principle. It is widely recognisedwwwwwwwwwwwwwwwwwwwwwww that for a game to be effective as a learning tool, its objectives and rules must be clear and straightforward. This clarity ensures that learners can easily understand what is expected of them, thereby reducing cognitive load and allowing them to focus on the application and understanding of new knowledge.



Expanding on the importance of design principles in educational gaming, the concept of flow theory offers a compelling framework for understanding how game design can enhance learning outcomes. Flow theory suggests a model in which the dynamics between goal clarity, learner concentration, and the effectiveness of learning are influenced by the characteristics of the learning environment. According to this theory, an optimal learning experience is achieved when there is a harmonious balance between the challenges presented by the game and the skill levels of the learner. This balance promotes a state of flow, a deeply immersive and focused state of engagement, where learners are more likely to absorb information effectively and develop new skills. The theory emphasises that the type of learning environment—whether it be competitive, collaborative, or exploratory—plays a crucial role in moderating the relationship between the clarity of goals, the concentration of the learner, and the overall learning effectiveness, highlighting the need for thoughtful game design in educational contexts.

This gives us a good example of how understanding objectives clearly can lead to much more balanced and engaging game. Without that a developer is much more likely to create a game like the infamous ET on the Atari or Superman 64
 
\break
\section{Sources}
\href{https://www.reddit.com/r/software/comments/ls6piz/good_software_for_designing_a_2d_platformer/}{Necessary expertise level for successful participation and performance of a game} \\
\href{https://www.argentics.io/how-vital-is-well-designed-ui-for-a-positive-game-user-experience}{Clarity in user interface design} \\
\href{https://www.researchgate.net/publication/263426149_Relationships_between_Goal_Clarity_Concentration_and_Learning_Effectiveness_when_Playing_Serious_Games}{Clarity in user interface design} \\
\href{https://marvelapp.com/blog/what-are-the-benefits-of-prototyping/}{Benefits of prototyping} \\
\href{https://www.design2market.co.uk/academy/what-is-a-prototype/}{Benefits of prototyping} \\
\href{https://ideafoster.com/en/benefits-prototyping/}{Benefits of prototyping} \\
\href{https://retrostylegames.com/blog/game-user-interface-design-examples/}{Clarity in user interface design} \\ 
\end{document}