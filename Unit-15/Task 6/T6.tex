\documentclass{article}
\usepackage{graphicx}
\usepackage{hyperref}
\hypersetup{
    colorlinks=true,
    linkcolor=blue,
    filecolor=magenta,
    urlcolor=cyan,
    pdftitle={Overleaf Example},
    pdfpagemode=FullScreen,
}
\usepackage{array}
\usepackage{titlesec}
\usepackage{geometry}
 \geometry{
 a4paper,
 total={170mm,257mm},
 left=20mm,
 top=20mm,
 }
\graphicspath{ {./images/} }

\begin{document}

\title{Unit 15 M3}
\author{Chris}
\date{}
\maketitle

\section{Notes}
Talk about 


\section{(P6)Present Prototype and Obtain Feedback}
*insert video of the game*
What we should expect to see in this video is that player is able to move freely and anywhere on the screen. You should also be able safe spawn in starting up the world.
The character is the player controlled ninija, this mets the rquirement "The game must have one identifiable character represented by a sprite that the user controls". This reflects the designs as we can see the in the video with the same character from our desigm documentation on the top. It was disigned so that that our target audience(young teenagers and childer)

The hud is a key comconept on the screen since it was is what the user sees all thtime
From the video, we can see that here is a visible hud with health, this can be related back to our design rationale in regards to the colour choice for the healt bar. These have meet the requirements that the main character must achieve poiints whether ethrough a health counter. THe HUD is clear so that the intended audience is more ingaged and can easily read it. IN a more developed version of the game it is recommended that pick ups can increase the health bar and this would lead to more of the target audience playing the game and more money for hte 

\textbf{The traps}



\section{(M3)Make changes to the design \& prototype based on feedback}

\subsection{analyse the results}
We have hit the requirements 
We have hit hte requirements by seeing what was outlined in the proposed document and making sure we hit the deadlines. The requiremens where clearly layed out and that made it easy.
The changes the game our client wants our is that the game shouold be more advanced movement and adding more difficult puzles for the player to 

Having more difficult playing means that game is more enganing to the player and there will hopefully be better advertising opertions. 
Ways of improving diffuclty
\begin{enumerate}
	\item Adding new different types of enemeys
	\item Adding weapons to help defeat the enemeys
	\item Making the puzzle more diffuclt/advance e:g adding items to help make the puzzles more advance 
	\item Increasing the diffuclty of platform by adding more trap, movving platforms and fake platforms
\end{enumerate}

Gauging diffculty is hard since it verys between each player and some players might complaing about hte 

*insert new game concept*

\section{(D2)Evaluate the game design and prototype}
The game design 






\end{document}
