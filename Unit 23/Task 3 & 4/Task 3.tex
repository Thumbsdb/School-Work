\documentclass{article}
\usepackage{graphicx} % Required for inserting images
\usepackage[colorlinks=true, allcolors=blue]{hyperref}
\usepackage{titlesec}
\usepackage{geometry}
\geometry{
a4paper,
total={170mm,257mm},
left=20mm,
top=20mm,
}

\title{Unit 23}
\author{Chris Hadden}
\date{}

\begin{document}

\maketitle

\section{(M1)Describe the drawbacks of an identified cognitive computing application}

\subsection{Cost}
Looking at the cost implications of adopting cognitive computing technology it may initially seem that they might be minimal due to the ease of understanding and implementing this technology. However, there are several factors to consider that could influence the costs associated with cognitive computing:

1. Development and Setup Costs: Even if the concepts behind cognitive computing are straightforward, the development and deployment of these systems can be complex and costly. Developing algorithms that can process and analyze large datasets, and ensuring they integrate well with existing systems, requires significant investment in terms of both time and money.

2. Training Costs: Cognitive computing systems typically require substantial data to learn and make accurate predictions. Preparing this data, and training staff to use and maintain these systems, can be resource-intensive.

3. Hardware and Infrastructure: Depending on the scale of the cognitive computing implementation, substantial hardware or cloud infrastructure might be necessary, which can be expensive.

4. Maintenance and Updates: Over time, cognitive computing systems need to be updated and maintained to handle new data, adjust to changes in the environment, and correct any errors or inefficiencies. This ongoing maintenance can be a significant cost.

5. Scalability: Scaling cognitive computing solutions to handle larger datasets or additional tasks can require more resources, potentially increasing costs.

While cognitive computing technologies can offer significant advantages, such as increased efficiency, better data analysis, and decision-making capabilities, the costs associated with them are not solely dependent on their conceptual simplicity. Careful planning and analysis are required to understand the total cost of ownership and to ensure that the benefits justify the investments.



\subsection{Limited analysis of risk}
Assessing risk accurately when implementing cognitive computing in a project can indeed be challenging, particularly due to the complexity and unpredictability inherent in cognitive computing systems. A few key points to consider when dealing with risk analysis in cognitive computing projects are:

1. Complexity and Predictability: Cognitive computing systems, which often employ machine learning and artificial intelligence, can exhibit behaviors that are difficult to predict or explain, even to the developers. This unpredictability can make it challenging to fully assess all potential risks.

2. Data Dependency: The effectiveness and behavior of cognitive computing systems heavily rely on the data they are trained on. Inaccurate, biased, or insufficient data can lead to unexpected outcomes, which in turn can introduce risks that are hard to foresee.

3. Human Oversight: While these systems are designed to operate automatically, human oversight remains crucial. Misunderstandings or misinterpretations of the system's functioning or outputs by human operators can lead to errors or oversight, adding another layer of risk.

4. Continual Learning: Many cognitive computing systems are designed to learn continuously from new data. While this is beneficial for improving system performance over time, it also means that the system's behavior can change in ways that might not have been anticipated at the outset.

5. Security and Privacy Concerns: The integration of cognitive computing into business processes often involves handling sensitive or proprietary data. Ensuring the security and privacy of this data against breaches or leaks is a critical risk that needs thorough assessment.

To better manage these risks, we should consider the following approaches:

- Thorough Testing and Validation: Before fully integrating a cognitive computing system into your business processes, conduct extensive testing and validation to understand how it behaves in different scenarios.
  
  - Robust Data Governance: Implement strong data management practices to ensure the quality and integrity of the data used by your cognitive computing systems.

  - Regular Audits and Updates: Regularly audit the performance and security of the system to catch and mitigate risks early. Keep the system updated to adapt to new threats and changes in the business environment.

  - Transparency and Documentation: Maintain detailed documentation of the system's design, capabilities, and limitations. This can help ensure that all stakeholders have a realistic understanding of what the system can and cannot do.

  - Risk Mitigation Strategies: Develop and implement strategies specifically aimed at mitigating the risks associated with cognitive computing, such as fallback procedures and human oversight mechanisms.

  Understanding and managing these risks is critical not only for the success of your project but also for maintaining trust and safety in the deployment of cognitive computing technologies.

\subsection{Security}

The security of our project is very imporant to us because without it we wouldn't have  trust from our users and we could get into legal troubles.
Cognitive computing, while powerful and beneficial for enhancing various business processes and decision-making, does introduce specific security drawbacks and challenges. Here are some of the key security concerns associated with cognitive computing systems:

\begin{enumerate}
	\item Data Privacy: Cognitive computing systems often require access to vast amounts of data, including potentially sensitive information. Managing and protecting this data to ensure privacy can be challenging, especially given the diverse sources and types of data that may be integrated.

	\item Vulnerability to Attacks: Cognitive computing systems, particularly those based on machine learning and AI, can be vulnerable to specific types of attacks, such as adversarial attacks. These are techniques that subtly manipulate input data in ways that are intended to cause the system to malfunction, often without detection.

	\item System Complexity: The complexity of cognitive computing systems can make them difficult to secure. Understanding all the potential vulnerabilities within a complex system can be daunting, and securing it requires continuous, expert attention and resources.

	\item Bias and Error Propagation: Security flaws can also arise from the biases present in the training data or the algorithms themselves. If a cognitive computing system develops or propagates bias, it can lead to incorrect or harmful decisions, affecting the integrity and trustworthiness of the system.

	\item Lack of Transparency: Many cognitive computing algorithms, especially those involving deep learning, are often described as "black boxes" because their decision-making processes are not easily understandable by humans. This opacity can make it difficult to diagnose and fix security vulnerabilities.

	\item Dependency on External Data and Services: Cognitive computing systems often depend on external data sources and third-party services for their operations. Each external connection introduces potential vulnerabilities and increases the attack surface of the system.

	\item Compliance Risks: Ensuring that cognitive computing systems comply with relevant laws and regulations, especially those concerning data protection and privacy, can be challenging. Non-compliance can lead to legal and financial repercussions as well as damage to reputation.
\end{enumerate}

Addressing these security challenges requires a robust security framework tailored to the specific needs of cognitive computing technologies. This framework should include rigorous data protection measures, continuous vulnerability assessments, and proactive threat management strategies to mitigate the unique risks these systems present.


\subsection{Adoption by enterprises}
The 4 Biggest Barries To adoption by enterprises are:
\begin{enumerate}
	\item Cultural barrier 
	\item Fear
	\item Shortage of talent
	\item Lack of a strategic approach to AI adoption
\subsection{Change Management}
Implementing cognitive computing within an organization often necessitates significant changes in management approaches. While these changes can lead to improved efficiency and decision-making, they also come with certain drawbacks. Here are some key challenges related to change management when introducing cognitive computing:

1. Resistance to Change: One of the most common challenges in implementing new technologies like cognitive computing is resistance from employees. This resistance can stem from fear of job displacement, a lack of understanding of the new technology, or discomfort with changing long-established workflows.

2. Training and Skill Gaps: Cognitive computing systems often require new skills and knowledge. There can be a substantial skill gap among existing employees, necessitating extensive training and potentially the hiring of new talent, which can be costly and time-consuming.

3. Integration Challenges: Integrating cognitive computing into existing IT systems and business processes can be complex and disruptive. Technical issues, compatibility problems, and the need to redesign workflows can lead to significant initial disruptions.

4. Cost Implications: The cost of acquiring, implementing, and maintaining cognitive computing solutions can be significant. Aside from direct costs, indirect costs such as training, system downtime during implementation, and increased support needs can add up.

5. Cultural Shifts: Cognitive computing often requires a shift in organizational culture towards more data-driven decision-making. This shift can be difficult to achieve, especially in organizations where decision-making has traditionally been intuitive or hierarchical.

6. Management Overhead: The need to manage and oversee the deployment and integration of cognitive computing can place additional strain on managers. This includes managing the change process, overseeing the technical implementation, and ensuring continuous improvement and adaptation of the systems.

7. Uncertainty and Risk: With any new technology, there is inherent risk and uncertainty, including the potential for project failure. Cognitive computing projects can be particularly risky due to their complexity and the critical nature of the data they handle.

8. Expectation Management: There can be a mismatch between expectations and reality in terms of the capabilities and immediate benefits of cognitive computing. Overcoming inflated expectations and demonstrating tangible benefits can be challenging.

To address these challenges, it's important for organizations to adopt a comprehensive change management strategy that includes clear communication, thorough training programs, and phased implementation. This strategy should also include strong leadership support and engagement strategies to help ease the transition, manage resistance, and align the technology with business goals.

\subsection{Lengthy Development Cycles}
The development cycles for cognitive computing systems can indeed be lengthy and complex. This extended timeline is influenced by several factors that are inherent to the nature of the technology and the tasks it is designed to perform. Here are some key reasons why cognitive computing development cycles can be lengthy:

1. **Complex System Design**: Cognitive computing systems are complex by nature. They involve advanced algorithms that need to simulate human thought processes to a degree. Designing these algorithms—and ensuring they perform correctly—takes significant time and expertise.

		2. **Data Acquisition and Preparation**: These systems rely heavily on data to learn and make decisions. Acquiring high-quality, relevant data and then preparing it for use (which includes cleaning, labeling, and structuring) can be a very time-consuming process.

		3. **Training and Testing**: Once a cognitive system is designed and the data is ready, the system must be trained. This training process can take a significant amount of time, especially as it often involves iterative tuning of parameters to optimize performance. Following training, extensive testing is required to ensure the system behaves as expected under various conditions.

		4. **Integration Challenges**: Integrating cognitive computing solutions into existing IT infrastructure and business processes can extend development cycles. These systems often need to interface with various other systems and technologies, requiring additional time for integration, testing, and deployment.

		5. **Regulatory and Compliance Issues**: Depending on the industry and application, cognitive computing systems may need to comply with specific regulatory requirements. Navigating these requirements and ensuring compliance can add to the development time.

		6. **Iterative Improvement and Scalability**: Cognitive systems often start with a pilot phase, which is then scaled and refined based on initial outcomes. This iterative process of refinement and scaling to ensure the system is accurate, efficient, and reliable extends the development cycle.

		7. **Ethical Considerations and Bias Mitigation**: Addressing ethical considerations and mitigating biases in cognitive computing (which learns from historical data that may itself be biased) requires careful design and continuous monitoring. This process can prolong the development phase as teams work to ensure the system’s fairness and transparency.

		To manage the lengthy development cycles, organizations can adopt agile methodologies, breaking the project into smaller, manageable parts that can be developed, tested, and rolled out incrementally. This approach not only helps in managing the complexity and scope of development but also in identifying and mitigating risks early in the development process. Additionally, leveraging automated tools for data handling and testing, and collaborating with experienced partners can also help reduce time and improve efficiency in development cycles.


\end{document}
