\documentclass{article}
\usepackage{graphicx} % Required for inserting images
\usepackage[colorlinks=true, allcolors=blue]{hyperref}
\usepackage[square,numbers]{natbib}
\usepackage{titlesec}
\usepackage{geometry}
\geometry{
 a4paper,
 total={170mm,257mm},
 left=20mm,
 top=20mm,
}

\title{Unit 23}
\author{Chris Hadden}
\date{}
\bibliographystyle{abbrvnat}

\begin{document}

\maketitle

\section{Purpose}
The primary goal of cognitive computing is to create computer systems capable of solving complex problems that typically require human cognition. This involves using advanced techniques such as machine learning, neural networks, natural language processing, computer vision, and others.

Cognitive computing solutions differ from traditional programmed systems in their ability to analyze vast amounts of unstructured data from various sources and uncover patterns and insights. They can interpret text, images, and speech, making connections across different data types. Over time, they continuously learn from interactions and experiences. As said in \cite{transform} "the use of AI in clinical decision-making is based on its ability to analyze a vast amount of patient data and identify patterns that can accurately predict clinical outcomes and recommend appropriate treatments"

This ability to emulate the natural learning process makes cognitive computing ideal for fields like healthcare, finance, and customer service, where large volumes of complex data need to be analyzed to find solutions.

\section{How Will It Achieve Its Purpose}
To achieve their purpose, cognitive computing systems must have the following attributes:

\begin{enumerate}
	\item \textbf{Adaptive:} Systems must be flexible enough to learn as information changes and goals evolve. They must digest dynamic data in real-time and adjust as the data and environments change.
	\item \textbf{Interactive:} Human-computer interaction is a critical component in cognitive systems. Users must be able to interact with cognitive machines and define their needs as they change. The technologies must also be able to interact with other processors, devices, and cloud platforms.
	\item \textbf{Iterative and Stateful:} Cognitive computing technologies can ask questions and pull in additional data to identify or clarify a problem. They must be stateful, keeping information about similar situations that have occurred previously.
	\item \textbf{Contextual:} Understanding context is critical in thought processes. Cognitive systems must understand, identify, and mine contextual data such as syntax, time, location, domain, user requirements, user profiles, tasks, and goals. The systems can draw on multiple sources of information, including structured and unstructured data, and visual, auditory, and sensor data.\cite{context}
\end{enumerate}

\section{How Would Cognitive Computing Help}
\begin{enumerate}
	\item Talking with more customers
	\item More accurate data analysis
	\item Leaner and more efficient business processes
	\item Improved customer interaction
	\item Increased employee productivity
\end{enumerate}

\break

\bibliography{bibliography}

\end{document}
