\documentclass{article}
\usepackage{graphicx} % Required for inserting images
\usepackage[colorlinks=true, allcolors=blue]{hyperref}
\usepackage{titlesec}
\usepackage{geometry}
\geometry{
a4paper,
total={170mm,257mm},
left=20mm,
top=20mm,
}

\title{Unit 23}
\author{Chris Hadden}
\date{}

\begin{document}

\maketitle

\section{M2: Minimise the social,moral and ehtical implications of your propose solution}

People can be retrained, and seeing as the chatbot can talk to patents when they are waiting to see a PG, this will improve the social aspects as the GP will not need to spend time doing that diagnosis and can spend more time with a patient to come up with a solution


AI-based diagnostic systems, while highly efficient, can occasionally produce erroneous results or 'hallucinate' illnesses. To mitigate these risks, it is essential to have trained personnel review and verify the diagnoses provided by the AI. Furthermore, to prevent the inadvertent leakage of patient information, AI systems should not continuously learn from new patient data once deployed. Instead, periodic updates and training should be conducted in a controlled environment using de-identified data. Additionally, patient acceptance of AI technologies can vary; some may prefer direct interaction with human healthcare providers. To accommodate these preferences, it should be possible for patients to opt-out of interacting with the AI system and choose traditional consultations instead. These measures will help ensure the AI's reliability, protect patient privacy, and maintain trust in healthcare services.


Cognitive computing systems require substantial data inputs to perform effectively, often including sensitive personal information like medical records or financial data. To safeguard privacy, it is imperative that this data is stored securely and that robust measures are implemented to prevent unauthorized access. Ensuring the integrity and confidentiality of this data is crucial for maintaining trust and compliance with relevant privacy laws and regulations.

\section{Reducing redundancies}
Reducing the impacts of the social,moral and ethical impacts of our solution are  
Our AI technology will assist in diagnosing our patients, enhancing the precision and efficiency of our healthcare services. We will retrain certain members of our staff to support patient care directly, ensuring a continued focus on personalized service. Additionally, other staff members will be transitioned into roles that leverage their skills more effectively and align with our commitment to improving healthcare outcomes.



\section{Supporting adoption}
Adopting our software may present initial ethical and social challenges; however, once integrated, it promises substantial benefits for all stakeholders involved. Our commitment to addressing these concerns is steadfast, ensuring that the deployment of our technology not only meets regulatory standards but also aligns with the highest ethical principles. We are dedicated to fostering trust and transparency, thereby facilitating widespread acceptance and maximizing the positive impact on our healthcare system.


\section{Build trust in cognitive system}
A trial period is pivotal for integrating our new AI technology into healthcare systems, as it ensures that all users are thoroughly trained to harness the full potential of the AI. This initial phase is crucial for the technology’s long-term success and effectiveness. Although training the staff presents challenges, it is indispensable for ensuring proper usage and sustained operation of the AI. Adequate education and support during this period are fundamental to achieving these goals."


\end{document}
