\documentclass{article}
\usepackage{graphicx} % Required for inserting images
\usepackage[colorlinks=true, allcolors=blue]{hyperref}
\usepackage{titlesec}
\usepackage[square,numbers]{natbib}
\usepackage{geometry}
\geometry{
a4paper,
total={170mm,257mm},
left=20mm,
top=20mm,
}

\bibliographystyle{dinat}

\title{Unit 23}
\author{Chris Hadden}
\date{}

\begin{document}

\maketitle
\section{D1 Evaluate the potential repurposing of your solution to other business sectors}
In this report we will outline how our Patient Diagnosis chatbot that we defined in Task 4 P3 could be used in other technological areas so that we can get the most value from our own technology.
\smallbreak
\section{Assessing prior art}
Here we are going to report on our research to find other implementations of diagnosis chatbots to see if other development teams have discovered repurposed implementations and if we can do the same with ours.
\smallbreak
A general point about GP chat points has been made by Noel Kennedy in that "A chatbot designed for the Healthcare industry to answer calls from the general public to
their GP greatly improves the efficiency of obtaining appointments, getting sound medical advice and reduces the workload on the staff in the GP’s surgery. Would this same technology be of use in the retail or finance sector to reduce call waiting times?" \cite{Noel}

This is a fair point and may point us in a lucrative area of research.
\smallbreak

\subsection{Comparing Physician and Artificial Intelligence Chatbot Responses to Patient Questions Posted to a Public Social Media Forum}
In the research paper Comparing Physician and Artificial Intelligence Chatbot Responses to Patient Questions Posted to a Public Social Media Forum\cite{bedside} one of the conclusions they came up with was that "a chatbot generated quality and empathetic responses to patient questions posed in an online forum" and "Further exploration of this technology is warranted in clinical settings, such as using chatbot to draft responses that physicians could then edit". In essence they are saying that the chatbot had a great bedside manner. This could lead us to thinking about using the chatbot as a therapist of some kind.


\subsection{Bookings}
One of the main uses of out chatbot was not just diagnosis but also possibly to make a booking to see GP. Looking in to chatbots that can make bookings we have found a plethora of businesses that have already been setup to do just this.\cite{futr}\cite{booking}\cite{velma} The example from booking.com is interesting as well as they mention "In September 2017, Finnair launched its first chatbot via Facebook Messenger. Nicknamed Finn, it can sell flights, calculate how much baggage a passenger can take and respond to questions." All of this is very advanced and shows that out chatbot could take on quite a lot.

\subsection{Exapnding on diagnosis}
The main use of our chatbot was to help with diagnosis of the patient's issues with a view to making the GP's work quicker and easier. This is a perfect use of a chatbot in that it augments the GPs work and allows them to assess more patients, especially as the chatbot's work can be done while waiting for the GP. 
This diagnosis facility could be used in other areas where a client knows what the symptoms of a problem are and need help to diagnose what the actual problem is. This is sometimes called Root Cause Analysis, and there are a number of chatbots that can already do this. \cite{gyan}\cite{devops}

The evaluation should show that the
learner has made a qualitative judgements taking into account different factors and using available knowledge and evidence



\bibliography{bibliography}
\end{document}
s