\documentclass{article}

\usepackage{graphicx}

\usepackage{hyperref}
\hypersetup{
    colorlinks=true,
    linkcolor=blue,
    filecolor=magenta,      
    urlcolor=cyan,
    pdftitle={Overleaf Example},
    pdfpagemode=FullScreen,
}
\usepackage{titlesec}
\usepackage{geometry}
 \geometry{
 a4paper,
 total={170mm,257mm},
 left=20mm,
 top=20mm,
 }

\title{Unit 15 P2}
\author{Chris}
\date{}

\begin{document}

\maketitle
\tableofcontents
\break

\section{P3 Create a design for an identified game concept}

\subsection{Game element}
\subsubsection{navigation}
The navigation in the game is done by providing important information to the player to the player by either using a Heads Up Display(HUD) for example compasses,maps or arrow signs. Both of these methods are helpful in games and their use depends on game's genre and how the game designers want the player to experience the game.

The HUD said by Greg Wilson in the article \textit{Off With Their HUDs} described HUD as "A collection of persistent onscreen elements whose purpose is to indicate player status. HUD elements can be use to show, among many other things, how much health the player has, in which direction the player is heading, or where the player ranks in a race" 



\subsubsection{scoring}
Scoring in games are key component of game mechanics and it provides a mechanic where the players get rewarded with point value whenever they accomplish a task in the game.  

\subsubsection{movement}
The movement for the game is

\subsubsection{interaction/controls}
The interaction/controls 


\subsubsection{conveying information}



\subsubsection{sound}

\subsubsection{levels}

\subsubsection{enemies}

\subsubsection{problem solving}

\subsection{Interface design}
\subsubsection{layout}

\subsubsection{colour palette}

\subsubsection{text styles}

\subsubsection{sound}

\subsubsection{stage/scene}

\subsubsection{actions}

\subsection{Character generation}
\subsubsection{bitmaps}

\subsubsection{wire frame}

\section{P4 Produce a logic structure for the identified game concept}
include diagrams to give evidence about the structure for the game. Give clear definition of objectives of the game. Flow chart showing the flow of the game through single or multiple layers with single or multiple players.

Include visualisation or written planetary designs or a combination of both and including alternatives, together with diagrams such as flowcharts.


\section{M2 Prepare alternative interface designs for the identified game concept}
Prepare alternative interface designs to the one identified in P3. The alternative designs must contain enough detail to enable them to be understood by a third party. Evidence can be extension of P3 and be presented as additional visualisations or written explanatory designs, or a combination of both.




\section{D1 Justify the design rational for the identified game concept}
Justify the designs choices and explain why they are suitable for the identified audience and purpose of the game concept. Evidence can be an extension of P3 and M2, and can be addition to the design documentation, a presentation or a report, but should reference the designs submitted.



\end{document}
