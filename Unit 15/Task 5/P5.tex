\documentclass{article}
\usepackage{graphicx}
\usepackage{hyperref}
\hypersetup{
    colorlinks=true,
    linkcolor=blue,
    filecolor=magenta,      
    urlcolor=cyan,
    pdftitle={Overleaf Example},
    pdfpagemode=FullScreen,
}
\usepackage{titlesec}
\usepackage{geometry}
 \geometry{
 a4paper,
 total={170mm,257mm},
 left=20mm,
 top=20mm,
 }

\title{Unit 15 M2}
\author{Chris}
\date{}

\begin{document}

Build \& test a prototype – P5
You are required to understand different ways that a prototype can be produced using bespoke
software for game development.

You are required to build a prototype for the game concept, whole demonstrating demonstrate
these programming techniques: variables/constants

\section{strings}
The strings in the game are we use are for the health

\section{inputs}
The inputs of our game are: Clicking on the main menu and movement, collecting items the timer and score


\section{outputs}
The outputs are the screen and if we have controller support 

\section{sequence}
The sequences are in the game 

\section{selection}
The selections are 

\section{iteration (e.g. counting/conditional)}
The counting  

\section{subroutines (e.g. functions/procedures)}
The subroutines 

\section{conditions}
The conditions are

\section{counting}
The counting done in this game is the score and you scan c

\section{totalling}
The totalling is the score counter this added 

\section{data structures (e.g. arrays/lists)}


\section{file handling}


\section{maintainable code}


\section{libraries}

Your prototype could be a particular level or module of the full game concept. Evidence of the
prototype could be presented in the form of annotated screenshots or as a video of a working
prototype, or the prototype itself.
You are also required to plan and carry out testing of their prototype. Testing must be thorough
enough to prove the prototype is functional and meets requirements. Classic testing techniques
include:

test plans
\section{test data}
The test data for the game will be
\begin{enumerate}
	\item Testing the hud
	\item Testing the items work
	\item Testing the levels that they can transfer
	\item Collision works
	\item Falling platform works
	\item Moving platform
\end{enumerate}

\section{black box}
The box testing takes advantage of extensive knowledge of an application’s internals to develop highly-targeted test cases. Examples of tests that might be performed during white box testing includeWhat is black box testing is a method of software testing that examines the functionality of an application without peering into its internal structures or workings. This method of test can be applied virtually to every level of software testing: unit, integration, system and acceptance. Black-box testing is also used as a method in penetration testing, where an ethical hacker simulates an external hacking or cyber warfare attack with no knowledge of the system being attacked. 

Black box testing involves testing a system with no prior knowledge of its internal workings. A tester provides an input, and observes the output generated by the system under test. This makes it possible to identify how the system responds to expected and unexpected user actions, its response time, usability issues and reliability issues.

Advantages of Black Box Testing

The tester does not need to have more functional knowledge or programming skills to implement the Black Box Testing.
It is efficient for implementing the tests in the larger system.
Tests are executed from the user’s or client’s point of view.
Test cases are easily reproducible.
It is used to find the ambiguity and contradictions in the functional specifications.

Disadvantages of Black Box Testing

There is a possibility of repeating the same tests while implementing the testing process.
Without clear functional specifications, test cases are difficult to implement.
It is difficult to execute the test cases because of complex inputs at different stages of testing.
Sometimes, the reason for the test failure cannot be detected.
Some programs in the application are not tested.
It does not reveal the errors in the control structure.
Working with a large sample space of inputs can be exhaustive and consumes a lot of time.


\section{white box}
White box testing takes advantage of extensive knowledge of an application’s internals to develop highly-targeted test cases. Examples of tests that might be performed during white box testing include
\section{alpha}
\section{beta}
\section{user testing}
All this work must be in your own words \& provide a comprehensive record of the development
and testing of your prototype.
16
It is important that you correctly reference all sources used, following appropriate conventions.

\end{document}
