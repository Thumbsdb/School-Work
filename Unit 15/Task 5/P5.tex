ss{article}
\usepackage{graphicx}
\usepackage{hyperref}
\hypersetup{
    colorlinks=true,
    linkcolor=blue,
    filecolor=magenta,      
    urlcolor=cyan,
    pdftitle={Overleaf Example},
    pdfpagemode=FullScreen,
}
\usepackage{titlesec}
\usepackage{geometry}
 \geometry{
 a4paper,
 total={170mm,257mm},
 left=20mm,
 top=20mm,
 }

\title{Unit 15 M2}
\author{Chris}
\date{}

\begin{document}

Build \& test a prototype – P5
You are required to understand different ways that a prototype can be produced using bespoke
software for game development.
You are required to build a prototype for the game concept, whole demonstrating demonstrate
these programming techniques:
variables/constants
\section{strings}

\section{inputs}
\section{outputs}
\section{sequence}
\section{selection}
\section{iteration (e.g. counting/conditional)}
\section{subroutines (e.g. functions/procedures)}
\section{conditions}
\section{counting}
\section{totalling}
\section{data structures (e.g. arrays/lists)}
\section{file handling}
\section{maintainable code}
\section{libraries}

Your prototype could be a particular level or module of the full game concept. Evidence of the
prototype could be presented in the form of annotated screenshots or as a video of a working
prototype, or the prototype itself.
You are also required to plan and carry out testing of their prototype. Testing must be thorough
enough to prove the prototype is functional and meets requirements. Classic testing techniques
include:

test plans
\section{test data}
\section{black box}
\section{white box}
\section{alpha}
\section{beta}
\section{user testing}
All this work must be in your own words \& provide a comprehensive record of the development
and testing of your prototype.
16
It is important that you correctly reference all sources used, following appropriate conventions.

\end{document}
