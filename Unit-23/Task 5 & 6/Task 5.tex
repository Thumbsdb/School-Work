\documentclass{article}
\usepackage{graphicx} % Required for inserting images
\usepackage[colorlinks=true, allcolors=blue]{hyperref}
\usepackage{titlesec}
\usepackage{geometry}
\geometry{
a4paper,
total={170mm,257mm},
left=20mm,
top=20mm,
}

\title{Unit 23}
\author{Chris Hadden}
\date{}

\begin{document}

\maketitle

\section{M2: Minimise the social,moral and ethical implications of your propose solution}

The biggest ethical issue is that people will be put out of a job and replaced with this chatbot. The best way to solve this is to repurpose anyone who is now free to spend more time with patients and their needs. People can be retrained, and seeing as the chatbot can talk to patents when they are waiting to see a GP, this will improve the social aspects as the GP will not need to spend time doing that diagnosis and can spend more time with a patient to come up with a solution


AI-based diagnostic systems can occasionally make made up results, what is know as 'hallucinate' them. To solve these risks, it is necessary to have trained personnel review and verify the diagnoses provided by the AI, again this is a function that people who are in danger of losing their job can perform. To prevent the inadvertent leakage of patient information cognitive AI should not continuously learn from new patient data once deployed. Instead some updates and training should be conducted in a controlled environment using de-identified data.
Patient acceptance of AI technologies can vary some may prefer direct interaction with human healthcare providers. To keep them happy it should be possible for patients to opt-out of interacting with the AI system and choose traditional receptionist instead. 
All this will help ensure the AI's reliability, protect patient privacy, and maintain trust in the healthcare services.


Cognitive computing AI requires a lot of data inputs to perform effectively often including sensitive personal information like medical records. To stop that leaking out it is important that this data is stored securely and that only tor right people are allowed access. Ensuring this will make sure the data is crucial for maintaining trust and compliance with privacy laws.

\section{Reducing redundancies}
Reducing the impacts of the social,moral and ethical impacts of our solution are important.
Our AI technology will assist in diagnosing our patients which will help precision and efficiency of our healthcare services. We will retrain certain members of our staff to support patient care directly, ensuring a continued focus on personalized service. Additionally, other staff members will be transitioned into roles that leverage their skills more effectively and align with our commitment to improving healthcare outcomes. Also as said before the AI technology needs to be monitored, this is a job that someone can be tasked to do as smaller part of any new activities.



\section{Supporting adoption}
Adopting our software may give initial ethical  challenges however once integrated it promises substantial benefits for all stakeholders involved. We need to make sure that our commitment to addressing these concerns for the customer is consistant and ensuring that the deployment of our technology not only meets regulatory standards but meets the best ethical principles. We are dedicated to creating trust and transparency to have a positive impact with the healthcare chatbot.


\section{Build trust in cognitive system}
A trial period is important for letting us try out the new AI technology in healthcare systems, as it makes sure that all users are thoroughly trained to use and get used to the AI. This  phase is important for the technology's long-term success and effectiveness. Training the staff presents challenges but it is indispensable to make sure they have proper usage and sustained operation of the AI. Teaching and support during that time are fundamental to making this a success.


\end{document}
