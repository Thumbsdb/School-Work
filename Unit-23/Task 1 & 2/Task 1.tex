\documentclass{article}
\usepackage{graphicx} % Required for inserting images
\usepackage[colorlinks=true, allcolors=blue]{hyperref}
\usepackage[square,numbers]{natbib}
\usepackage{titlesec}
\usepackage{geometry}
\geometry{
 a4paper,
 total={170mm,257mm},
 left=20mm,
 top=20mm,
}

\title{Unit 23}
\author{Chris Hadden}
\date{}
\bibliographystyle{abbrvnat}

\begin{document}

\maketitle

\section{Purpose}
The purpose of developing cognitive computing is ro create computer systems that can help people do their job. When we are doing this we are going to use machine learning, neural nets, and natural language processing.

Cognitive computing differs from programmed systems in their ability to process lots of data from many sources and uncover patterns and insights. These can all sorts of inputs which can see patterns from across different data. They should keep learning all the time. \cite{transform} "the use of AI in clinical decision-making is based on its ability to analyze a vast amount of patient data and identify patterns that can accurately predict clinical outcomes and recommend appropriate treatments"

As cognitive computing can read and write like people it makes cognitive computing ideal for fields like healthcare, finance, and customer service where lots of complex data need to be analyzed.

\section{How Will It Achieve Its Purpose}
Cognitive computing systems need to be able to do the following:

\begin{enumerate}
	\item Adaptiv Flexible to learn as information comes in.
	\item Interactive Users must be able to interact with cognitive machines and say what they want.
	\item Stateful They must be keeping information things that have occurred previously.
	\item Contextual Understanding context is critical in thought processes. Cognitive systems must understand, identify, and mine contextual data such as syntax, time, location, domain, user requirements, user profiles, tasks, and goals. The systems can draw on multiple sources of information, including structured and unstructured data, and visual, auditory, and sensor data.\cite{context}
\end{enumerate}

\section{How Would Cognitive Computing Help}
\begin{enumerate}
	\item Talking with more customers
	\item More accurate data analysis
	\item Efficient business processes
	\item Improved customer interaction
	\item Increased employee productivity
\end{enumerate}

\break

\bibliography{bibliography}

\end{document}
