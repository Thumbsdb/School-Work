\documentclass{article}
\usepackage{graphicx} % Required for inserting images
\graphicspath{ {./images/} }
\usepackage[square,numbers]{natbib}
\usepackage[letterpaper,top=2cm,bottom=2cm,left=3cm,right=3cm,marginparwidth=1.75cm]{geometry}
\usepackage{hyperref}

\bibliographystyle{dinat}


\begin{document}


\title{Unit 13}
\author{Chris Hadden}
\date{}
\maketitle

\section{P4 Legal and ethical restrictions on the use of social media as part of
digital marketing campaigns}

In Titanica Media we take great care in dealing with our legal obligations, because if we do not then there could be severe fines and we could even be disallowed from collecting social media data.
We will start this report by explaining why we need to take these issues seriously
\section{Social media scandals}
In 2018 the Facebook and Cambridge Analytica scandal broke. This involved Cambridge Analytica harvesting the personal data of millions of Facebook users without their consent\cite{cac}. This was a watershed moment for the public as is it exposed how their data was being used by companies.
In 2014 Facebook creates a quiz for people to find their personality type. The quiz collected the users data and their friends. It gathered data on 87 million people. Some of this was sent to Cambridge Analytica. 

what the exact legislation is that we need to work with

\section{Legislation}
\subsection{Data Protection Act 2018}
The Data Protection Act 2018 is a law in the UK that sets out how personal information should be handled to protect individuals' privacy. 
It builds aligns with the General Data Protection Regulation (GDPR) which is a comprehensive privacy regulation from the European Union.
It gives users of social media power over how their personal data is used and stored. 
The main concepts are
\begin{itemize}
    \item Personal Data Protection
    \item Control Over Personal Data
    \item Consent
    \item Transparency
    \item Data Use Limitation
    \item Data Protection Officers
    \item Penalties for Non-compliance
\end{itemize}

\bigskip

The main restrictions on our company that you need to be aware of are
\begin{itemize}
    \item Lawfulness, Fairness, and Transparency Data must be processed lawfully, fairly, and in a transparent manner. This means we must have a valid basis and user consent for processing personal data and must clearly inform individuals about how their data is used. This means that our business uses social media data we have to be compeletly sure that the data was sourced only from users who consented.   
    \item Purpose Limitation. We can only collect personal data for specified, explicit, and legitimate purposes. Any social media data we use can only be used for exactly what it was collected to do, and no more.
    \item Data Minimization. We should only collect personal data that is necessary for the purposes for which it is processed. We must rely on social media sites or who ever collected the data to ensure that only th necessary data was collected when it is passed to ut. The moral obligation is on the collector, which may be us.
    \item Accuracy. Personal data must be accurate and, where necessary, kept up to date. As a social media data processor we have the ethical imperitive to be able to delete or update users data if the request that.
    \item Storage Limitation. Personal data should be kept in a form that permits identification of data subjects for no longer than is necessary. We need to have a plan on how to store and delete social media data and an audit trail to prove we did what we said we would do.
    \item Integrity and Confidentiality. Personal data must be processed in a manner that ensures appropriate security of the data. Ethically we need to be able to show we handled social media data in a way that will not leak personal details.
\end{itemize}


\subsection{Copyright, Designs and Patents Act 1988}
The Copyright, Designs and Patents Act 1988 is a UK law that provides protection for creators of original works, such as writers, artists, and inventors.

It's main points are
\begin{itemize}
    \item Copyright. The law protects original works of authorship, including literary, dramatic, musical, and artistic works, plus sound recordings and films. It gives creators exclusive rights to use, reproduce, and distribute their works. This means that we can not use any images or media from social media or anywhere else with out the creator's consent. This can also apply to home photographs or any content output by a user.
    \item Designs. This protects the visual appearance or aesthetics of a product, provided the design is new and has individual character. This means we can not use well known brand designs in our own social media output.
    \item Patents. Patents protect new inventions, allowing inventors exclusive rights to make, use, sell, and import their inventions.

\end{itemize}

What this means for Titanica Media is that we cannot use copyrighted works, designs or patents without permission from the copyright holder, which may involve licensing agreements.


\subsection{The Privacy and Electronic Communications (EC Directive) Regulations 2003}

The Privacy and Electronic Communications Regulations (PECR), are regulations that focus primarily on protecting personal privacy in electronic communications.

It's main points are
\begin{itemize}
    \item Cookies and Similar Technologies. We must inform users about any cookies or  technologies that track information about users on websites. Users must also give their consent before these cookies can be used. This is a huge restriction in social media but as it is law we must comply with ensuring that users are aware of any tracking cookies, etc.
    \item Marketing Communications. PECR restricts sending electronic marketing messages without the recipient's prior consent. Any emails or posted marketing we send out has to have the explicit consent from the social media user.
    \item Security of Services. We must ensure that their services have adequate security measures to protect users' data from being accessed or altered without authorization. We have to make sure that we store social media users data securely as it is unethical to do otherwise.
    \item Traffic and Location Data. PECR requires companies to keep any data that can track users' locations confidential, unless users have given consent for their data to be used. If we do track social users then we need to have explicit consent from the users. 
\end{itemize}

For us this means that we need to ensure that we have users consent anytime we deal with their data or contact them.

\newpage
\section{Business policy and practice}
We have various predefined policies that will help out in these areas of legislation.

\subsection{Acceptable Use}
This outlines the rules and guidelines for using the systems and data of Titanica Media.
It has a \textbf{Purpose} and \textbf{Scope} that outlines why they exist and who it applies to.
It describes what \textbf{Legal} compliance we are bound to. Importantly it defines \textbf{Data Access} and what use we make of it.

\subsection{Social Media}
This policy is important for ensuring that the use of social media aligns with the our values and protects our reputation.
Again, it has a \textbf{Purpose} and \textbf{Scope} that outlines why they exist.
It outlines everyone \textbf{Responsibilities} in terms of accessing social media data and describes any \textbf{Prohibited Behavior}

\subsection{Recruitment Use}
This policy exists to make sure that our hiring practices are fair, transparent, and legally compliant.
It also has a \textbf{Purpose} and \textbf{Scope} that outlines why they exist.
It defines the \textbf{Recruitment Process} making sure there is no bias or requesting of personal information.
It ensures \textbf{Equality and Non-Discrimination} in the process and it ensures \textbf{Confidentiality and Security}
 

\subsection{Your responsibilities}
All these polices enforce our Ethical standpoint on user data and it's use. While we can make money for the company by selling on data, we typically do not do that and if we do then we have to comply with all legislation and have to get sign off from our Data Protection Officer.

By way of an example, if we do not comply then we could loose reputation and get fined like Meta recently was

In May last year Ireland's Data Protection Commission (DPC) investigated Meta Platform Ireland Limited ("Meta Ireland"). They fined the company €1.2 billion for not following proper privacy rules. The DPC said Meta Ireland sent personal data from Europe to the US without making sure it was kept private, which was against the rules while they were running their Facebook services. \cite{Meta}

\newpage
\section{Ethical and moral issues}
\subsection{Bias}
Bias can become a large issue when dealing with data on the scale of social media data. Bias can happen when querying peoples opinions and factors such as age, background or confirmation of humanity is checked. A YouGov study in 2024 found that 20\% of young adults in the US thought the holocaust was a myth. It turned out that the polling was flawed as the respondants were mostly people who filled out surveys to get in game currency, so they were putting any results in without thinking \cite{moreorless}
This type of bias can sway any data we collect as we may, as an example, inadvertantly just be asking young adults and getting skewed results that way. Also internally we should ask staff and others to declare any other conflicting interests they may have.


\subsection{Integrity}
We need to consider how we will use the social data we gather. If we get any unethical insights such as how people might get addicted to our product, or how to use negative opinion against competitors. If we use this information then we have lost our integrity. It is up to us to have processes to deal with data with integrity and lead by example.

\bibliography{bibliography}

\end{document}

