\documentclass{article}

% Language setting
% Replace `english' with e.g. `spanish' to change the document language


% Set page size and margins
% Replace `letterpaper' with `a4paper' for UK/EU standard size
\usepackage{geometry}
 \geometry{
 a4paper,
 total={170mm,257mm},
 left=20mm,
 top=20mm,
 }
%Packages
%Coloums

%\usepackage{tabularx}
%\usepackage{amsmath}
%\usepackage{graphicx}
\usepackage[colorlinks=true, allcolors=blue]{hyperref}
\usepackage{titlesec}

\title{Project Life Cycles}
\author{Chris Hadden}

\begin{document}

\maketitle

\tableofcontents

\section{(P3) Complete the documentation for the initiation phase for an identified project}

\subsection{The stakeholders}
Our game involves three key stakeholder groups: the client, the end-user, and the publisher. The client, as the primary stakeholder, is the reason for the game's creation. We are developing the game to feature and promote the client's foods and mascot, with the project being funded by them. Equally crucial are the end-users, whose engagement is essential for generating revenue and providing a platform for our client to advertise their offerings. Lastly, the publisher plays a vital role in ensuring the game reaches our end-users, making it possible for them to experience the game. Without the publisher, the end-user wouldn't know who we are.


\subsection{Clients and Target audience}
Our game is specifically designed for children and young teenagers, featuring straightforward gameplay and simple mechanics for ease of understanding. As a platformer, a genre well-known and widely embraced, its game mechanics are familiar, allowing players of all ages to easily recognise and engage with this type of game.

Our game's visual and interactive elements are tailored to captivate and retain the attention of a young audience, ensuring an enjoyable and accessible gaming experience. Furthermore, by incorporating universal platforming elements, we create a welcoming environment for both new and seasoned gamers, fostering a broad appeal across diverse age groups.

\subsection{The scope definition}
This project will have a focused scope, determined by the inclusion of only five levels in the game. This approach allows us to streamline the selection of products we intend to promote to the end-user for potential purchases. Each level will be themed around a specific food item, such as chocolate bamboo sticks for the first level and boba drinks for the second, among others. This targeted strategy aims to enhance the likelihood of end-users purchasing our client's food products and to advertise the client's business.

\subsection{The purposes}
The primary goal of this project is to promote and sell our client's food products. We plan to design each of the five game levels around various sweets offered by our client, featuring their mascot collecting these items. The game's difficulty will increase progressively with each level. Additionally, we aim to educate the end-user about the variety of foods our client offers, highlighting the unique qualities of each.

\subsection{The objectives}
The primary goal is to encourage end-users to purchase our client's food products. To achieve this, each level will be themed around a specific food item from our client's selection, with health boosts in the game corresponding to the food featured in that level. Upon completion, players will be rewarded with a lifetime supply of the client's food for their character.

To maintain player engagement, various smaller objectives will be integrated, such as defeating monsters to gain points and navigating obstacles that increase the level's difficulty for the end-users


\subsection{The deliverables}
The deliverable are the list of features that the user will be able to access once the project. These features are
\begin{enumerate}
    \item Five unique levels themed after foods
    \item Items themed after our clients food
    \item Promote the clients food
    \item Link to the clients website
\end{enumerate}
 

\subsection{The timescales}
The timescale of the project will be very short since there will only be 5 levels to this game. We except this project to be done in a very short period of time since making a 2D platformer is easy and there are loads of tools out there to make the game.

The initial stages of the project are going to be:
Talking to the our client and making sure everything is what the client meet and that it sells the foods.

\subsection{The structure}
The structure of the project will be quite flat. We will need engagement with the stake holders planned out.
In our team we will need
\begin{enumerate}
    \item A graphic designer
    \item Sounds effects designer
    \item Game engine designer
    \item Project lead
\end{enumerate}
Some of these roles be done by one person, or we could have multiple people hired for each role.

The structure to the game is quiet simple. It will have 5 levels which get more difficult for the player progresses thought the stage. Each level will show off the foods are client want to show off to our customers 

*chart
The game's structure is straightforward, consisting of five levels that increase in difficulty as the player advances through each stage. Each level is designed to showcase the food products our client wishes to promote to their customers.





\break
\section{(D2) Create a business case to support an identified project}

\subsection{Project justification}
In the current market landscape, there is a push for brands to establish their presence through new methods. Traditionally, brand promotion was achieved via conventional advertising mediums such as television and billboards. However, contemporary audiences are increasingly disengaging from traditional media formats, such as newspapers and linear television, presenting a unique challenge to marketers. The objective of this project is to explore and identify innovative advertising strategies that effectively engage customers, given the diminished efficacy of traditional channels.

\subsection{Risks}
The main risk in not doing anything is that the customer's brand will fall out of favour and either be forgotten about or seen as old fashioned and thus not appealing to a younger generation. 
The flip side to this risk is if a brand tries to reach out to a younger generation but does it in a patronising way then the brand could suffer by not only being seen in a bad light but also they may lose a lot of money that was spent on the advertising.

\subsection{Business benefits}
The benefits in reaching a younger generation are that the brand will stay relevant and interesting. Also having reached a younger customer there is a chance that they will stay brand aware for a long time. All of this will hopefully pay off in same way that traditional advertising does.

\subsection{Critical impact on the business}
The critical impact on the business will be that the business will not dissolve in to oblivion as their brand becomes less known about or respected. Stemming this consciousness flow is very important as it is possible for a branch to reach a point of no return when no amount of relaunches, etc will have any impact.
Getting this for of advertising correct though will pay off for the future of the company. Getting it right will keep the brand in customers heads and make the brand seem contemporary and aware of modern trends.

\break
\subsection{Solution 1}
\subsubsection{Overview}
The solution will involve hiring an internet advertising agency and developing an online advertising presence. An initial engagement will need to be had with the agency to describe how the customer sees it's branch at the moment and how they would like to be seen. 
The advertising agency will be able to come up with a number of solutions and we may want to select to meld of different options, such as adverts in social media, using advertising brokers to advertise on various related websites and possibly sponsorship of well known internet characters.
Hopefully they will be able to help devise what the advert contents should contain and drive the branch in the correct direction.

\subsubsection{Risks}
The first risk is that the action of advertising online itself could backfire. Many people do not like adverts online as they find them intrusive and are concerned that the advert track them though the internet.
The second risk is that the adverts may come over as patronising or the the brand may make themselves look like they do not understand modern paradigms. 
The third risk is that it may just not work. Some people block ads and some people are adept at scrolling past them. If too many people do that then the effort will have been wasted.
\subsubsection{Remedial actions}
It will be hard to counteract the impression that online advertising gives. In this case we would take advice from the advertising agency and let them take the lead on how invasive adverts are and where the adverts appear.
Again, we will have to takes the advertisers lead on ensuring that the adverts are not patronising, but to be sure we will need to conduct our own testing of the adverts before we launch, to find out what people in general feel about them.
There are means to counteract ad blockers and scrolling. We will need the advertising company to provide us with metrics about how many adverts actually appeared in front of people and which adverts had the best click through rates. At the very least, if the advert fails, we want to know it has failed.

\subsubsection{Associated costs}
The cost of this solution is entirely up to the client. With advertising they can spend as much or at little as they like, though typically there is a point of diminishing returns on top end spending.
A minimum spend would be expected to be in the region of a few thousand dollars, and on the top end anything over a few hundred thousand would be seen as excessive.

\subsubsection{Timescales}
The timescales involved again would depend on the advert and content. We would recommend a trial period with a simple advert which will be cost effective and also allow the company to determine if the adverts are having the desired effect

\subsubsection{Investment benefits}
Seeing as the initial recommendation is to have a small investment in the advertising being proposed, this will mean that the timescales are also short. This all in turn means that it should be quick and relatively cheap to determine if advertising is working or not. Based on that a further round of advertising could be booked to capitalise on the initial round of adverts, or not if they were deemed to not have worked.
Either way, the cost of failure is cheap and the pay off if the adverts work could be what sets the company in to a positive future.
This however comes with it's own risk, if not enough money is spent on the initial adverts, they may not have enough reach and will have appeared to have failed when they normally would not have. This may be insurmountable without spending a lot of money initially

\subsection{Solution 2}
\subsubsection{Overview}
This solution will involved the development of an interactive, online game that is based on the client's brand. This will not only engage customers but it will mean that for an extended period people will be exposed to the client's brand. 
The game itself should be already quite familiar as we want to have engagement without putting people off. Also while the game itself may not be of interest to everyone, advertising of the game will also act as advertising of the brand so there is a fallback built in to the advertising.

\subsubsection{Risks}
Development of a game is usually not a trivial undertaking and we will either need to hire the necessary people to develop the game or bring in a third party to develop it, at which point we are heavily relying on an external party.
It will take longer to develop a game than other types of advertising, this could make it more expensive in a number of ways.
Customers may be put off by the game and in turn find that the brand is confusing as they do not understand what the brand stands for.
There is recent trend for social media companies to demote posts which have an external link in them, such as a link to our game.

\subsubsection{Remedial actions}
Hiring contractors in house to develop the game will mean we have full control of what they are developing and will be able to track their work in-house, which will reduce overall risk.
The development of the game should be done in an agile manner so that we can play test it as it is being developed. The requirements for the game need to be very clear initially so that there's no risk of development an unsuitable game. The requirements for the UI should be developed in tandem with a UX company to make sure that the game is accessible for all users and will look modern.
The game should be tested on all supported platforms and an external play testing company should be hired to make sure that the game and UX experience is as smooth as possible.
We would need to pay for advertising on social media to ensure that links to our game are seen. The benefit of this is that it's extra advertising for the brand.

\subsubsection{Associated costs}
The main costs are likely to come from
\begin{enumerate}
    \item Hiring some developers
    \item Hiring play testers
    \item Hiring an advertising agency to advertise the game
    \item Hiring a Project lead
    \item Hiring a UX company
    \item Game hosting
\end{enumerate}
Given the timescales are likely to be in and around three to six months, the cost of development is likely to come out at about \$500000

\subsubsection{Timescales}
The lead time in development of a simple online game is typically three to six months. The advertising consolation can go on in parallel though we would need initial drafts of the game to show them.
The advertising then would typically run for two or three months, to be able reach the maximum number of people, but not saturate the brand

\subsubsection{Investment benefits}
This is an expensive proposition, but it has been proven that internet advertising has the greatest reach in today's world. Internet advertising is a mature market and it is hard to stand out without paying extortionate amounts of money. The pay off with creating a game is that after the initial advertising has been payed for people can come back to the game time and time again and this will cost the brand very little in terms of hosting costs and nothing else, as is proven by Sky TV's Beehive Bedlam.
While normal advertising has a short shelf life, this advertising has the capacity to stick in peoples consciousness and keep paying for itself in to the future.

\subsection{Solution 3}
\subsubsection{Overview}
Social media is arguably the largest social shift seen in decades. Many brands have brand ambassadors on multiple social platforms that represent the brand in question. The solution would involve having an active social presence on the main platforms. This would include creating videos on YouTube and TikTok, posting on Twitter and doing both on Instagram, etc

\subsubsection{Risks}
Social media can be fraught with risks. Mainly been seen as insincere and only being present to make money. It is also very easy to make social faux pas that can quickly cause groups to form on those platforms to try and cancel your brand.
Creating all the necessary content can be take a long time and be expensive.

\subsubsection{Remedial actions}
Who ever is hired to do the social media postings will need to be experienced in this area and also under contract not to disparage our brand.
Posting text is cheap and quick, this will save money. Making videos on your phone can be now be done relatively inexpensively and will also give the content more of a real feel.

\subsubsection{Associated costs}
The main cost will be hiring a social media manager. They need to be experience and to prove through their portfolio that we can trust them. 
The other costs will be hardware such as cameras.
The other costs will be the time involved with other team members to talk to video, podcasts and to write blogs.

\subsubsection{Timescales}
The will be an ongoing investment, but the initial setup will be relatively quick. It should take no more the three to six months to hire a social media manager.

\subsubsection{Investment benefits}
The brand will be seen as being on a par with other modern brands.
Also the brand will be in the middle of people's online life, which can be pervasive. 

\subsection{Recommendation}
Given all the solutions, at this point we would suggest solution 2. 
The reason for this is that
\begin{enumerate}
    \item It's a very time bound solution, once it is over it can still work for free
    \item It will make the brand look modern
    \item It will not suffer from unbound spending
    \item It is very controlled so the brand will know what they are getting
    \item The advertising for the game also acts as advertising for the brand while the game itself keeps the brand in the front of people's minds
    \item Traditional advertising is seen as not being effective so a more interactive approach should catch people's attention
\end{enumerate}
 
\end{document}