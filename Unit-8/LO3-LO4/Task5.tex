\documentclass{article}
\usepackage{graphicx}
\usepackage{hyperref}
\hypersetup{
    colorlinks=true,
    linkcolor=blue,
    filecolor=magenta,
    urlcolor=cyan,
    pdftitle={Overleaf Example},
    pdfpagemode=FullScreen,
}
\usepackage{titlesec}
\usepackage{geometry}
 \geometry{
 a4paper,
 total={170mm,257mm},
 left=20mm,
 top=20mm,
 }
\graphicspath{ {./images/} }

\begin{document}

\title{Unit 8 Task 5}
\author{Chris Hadden}
\date{}
\maketitle

\section{(P5) Follow the project plan and conduct a phase review of the identified project}
*
*REWORD THIS
*
*
*Add the game design into this document
*
Refer to how you used and followed your planning document
\subsection{Time}
All the activity will have been allocated a start and finish date before hand so that was easier to know what and when the task should be done. IF the task toke long than expect, the project manage will had to consider:
\begin{enumerate}
	\item Is  the task on a critical pathway? Is failing to meet the deadlines likely to result in delaying completion of the project
	\item How can it be brought back on track?
	\item Is there a case to spend additional resources to bring the activity to a timely completion or at least to reduce the delay>
\end{enumerate}

\subsection{Quality}
 The project manager has to monitor the quality of all the processes being carried out by his team:
 \begin{enumerate}
 	\item Are they in line with th procedures laid down?
	\item Are they ensuring that the deliverables are being created in a timely manner to the required standard?
\end{enumerate}

Quality management

A good quality management plan identifies and manages these quality standards.

Quality audit s throughout the project review the quality standards and the assumptions made at the start of the plan to ensure that the standards are being maintained, including under budget and to the time agreed


\subsection{Change}

The project manager needs to consider any proposed changes to the project, the reasons for the requests and the effect that change will have on the timeliness and costs contained in the plan.

The project plan should be updated and the changes approved with the project sponsor.

\subsection{Risks}

The risks identified in the plan need to be monitored. As the project develops, some risks may no longer be relevant and this is recorded on the plan. 

Other risks may arise, such as changes or additions to the deliverables being requested by the sponsor or the departure of a team member with specialist knowledge

\subsection{Issues}
Issues are not the same as risk. Here is why risk are reasons why risk and issues are different
\begin{enumerate}
	\item Risk can be anticipated
	\item Issues arise unexpectedly
\end{enumerate}

In this project as the project manager I must ensure that:
\begin{itemize}
	\item
\end{itemize}

\subsection{Communications}

\subsection{Acceptance}

A phase review
		\subsection{Is the project on schedule}
		\subsection{Is the project within budget}
		\subsection{Have the deliverables been produced and approved}
		\subsection{Have the risks been controlled and mitigated}
		\subsection{Have issues been resolved}
		\subsection{Is the project on track}
		\subsection{Should the project contriun}


\section{(D3) Prepare a project closure report based on the execution phase review of the identified projectK}

\end{document}
