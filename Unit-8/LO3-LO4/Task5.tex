\documentclass{article}
\usepackage{graphicx}
\usepackage{hyperref}
\hypersetup{
    colorlinks=true,
    linkcolor=blue,
    filecolor=magenta,
    urlcolor=cyan,
    pdftitle={Overleaf Example},
    pdfpagemode=FullScreen,
}
\usepackage{titlesec}
\usepackage{geometry}
 \geometry{
 a4paper,
 total={170mm,257mm},
 left=20mm,
 top=20mm,
 }
\graphicspath{ {./images/} }

\begin{document}

\title{Unit 8 Task 5}
\author{Chris Hadden}
\date{}
\maketitle

\section{(P5) Follow the project plan and conduct a phase review of the identified project}
*
*REWORD THIS
*
*
*Add the game design into this document
*
Refer to how you used and followed your planning document
\subsection{Time}
All the activity will have been allocated a start and finish date before hand so that was easier to know what and when the task should be done. IF the task toke long than expect, the project manage will had to consider:
\begin{enumerate}
	\item Is  the task on a critical pathway? Is failing to meet the deadlines likely to result in delaying completion of the project
	\item How can it be brought back on track?
	\item Is there a case to spend additional resources to bring the activity to a timely completion or at least to reduce the delay>
\end{enumerate}

\subsection{Quality}
 The project manager has to monitor the quality of all the processes being carried out by his team:
 \begin{enumerate}
 	\item Are they in line with th procedures laid down?
	\item Are they ensuring that the deliverables are being created in a timely manner to the required standard?
\end{enumerate}

Keeping to the plan that we designed in * insert doc * so keeping to makes it easy to keep to the quality under control. When we do make changes of our own we try to stick a strict plan so that quality is made and meet so that the client is happy with the project. Making sure the client is happy and pleased with the project is the purpose of a good quality product 

Quality management

A good quality management plan identifies and manages these quality standards.

Quality audit s throughout the project review the quality standards and the assumptions made at the start of the plan to ensure that the standards are being maintained, including under budget and to the time agreed


\subsection{Change}
Changes can be made in our games project is going hard since games project  

The project manager needs to consider any proposed changes to the project, the reasons for the requests and the effect that change will have on the timeliness and costs contained in the plan.

The project plan should be updated and the changes approved with the project sponsor.

\subsection{Risks}
The risk in this game have all been thought up a head of time to make sure that 
The risks identified in the plan need to be monitored. As the project develops, some risks may no longer be relevant and this is recorded on the plan. 

Other risks may arise, such as changes or additions to the deliverables being requested by the sponsor or the departure of a team member with specialist knowledge

\subsection{Issues}
Issues that can com ame as risk. He
Issues are not the same as risk. Here is why risk are reasons why risk and issues are different
\begin{enumerate}
	\item Risk can be anticipated
	\item Issues arise unexpectedly
\end{enumerate}

In this project as the project manager I must ensure that:
\begin{itemize}
	\item
\end{itemize}

\subsection{Communications}

\subsection{Acceptance}

Phase review for hte 

A phase review
		\subsection{Is the project on schedule}
		\subsection{Is the project within budget}
		\subsection{Have the deliverables been produced and approved}
		\subsection{Have the risks been controlled and mitigated}
		\subsection{Have issues been resolved}
		\subsection{Is the project on track}
		\subsection{Should the project contriun}


\section{(D3) Prepare a project closure report based on the execution phase review of the identified projectK}

\subsection{actions identified in the phase review that had to be addressed so that the project could be closed}

The actions that need to succesfully close the project are
\begin{enumerate}
	\item Get all the adverts out 
	\item Make sure the client is happy everything
	\item Make sure all the deliverables where met
	\item 
\end{enumerate}

\subsection{project summary}
The project summary is that 


\subsection{reason for project closure(has the project reachec its planned aend adate or is it being closed prematurely for some reason)}
The project was closed because we had done everything our client wanted and they had closed of the project on there end after a sucessful run. The project closed when the game no longer served it's purpose of selling and promoting our clients food/


\subsection{assessment of project performance(how well did the project perform against the business case)}
Assessment of the project performance went well since our clients since a 25\% incease in their sales after realsing our game out to the world. We went the projects that we has set, that where "encourage end-users to purchase our client's food products" and to have each level themed a=after a specific food item form our client's selection. 

The performace compared to the planned time 
When comparing our project against the outline cost and planned time we have successfully hit our goals of being on time and with in budget. Keeping in budget was hard but when the correct managers it was easy since with having a gantt chart showed us what task on on time and what are late. Since we where able to see this it made it easy for us to adjust other task to fit in with the time window and make sure task didn't exceed there limit. The gantt chart also helps with the budget aswell since time cost money and the moer time a task takes the more money we spent.

For the future there where areas we could have saved money. These areas are

	achievement of the project objectives, outputs and activities
	performance against planned time, resources and cost
	performance against planned savings and benefits

\subsection{lessons learned}
The lessons learned from this experince are that working with client saves everyone time and money invovled. Keeping on track is hard but rewarding task todo, it might cause it bit more stress involved but means that we all save our time and money which I feel is more important to save than going over time and cost.

\subsection{celebrating success}
Celebrating success is quite important since the team needs an end to the project so that they don't feel over worked and under stress all the time, so celebrating is all so a big part in the project.


\subsection{next steps}
\subsubsection{outstanding activities}
\subsubsection{remaining risks and issues}
\subsubsection{on-going dependencies}
\subsubsection{costs of on-going support}
\subsubsection{stakeholder communications}
\subsubsection{handover of asseets  and contracts}




Project closure report will be

Project closure execution 



\end{document}
