\documentclass{article}
\usepackage{graphicx}
\usepackage{hyperref}
\hypersetup{
    colorlinks=true,
    linkcolor=blue,
    filecolor=magenta,
    urlcolor=cyan,
    pdftitle={Overleaf Example},
    pdfpagemode=FullScreen,
}
\usepackage{titlesec}
\usepackage{geometry}
 \geometry{
 a4paper,
 total={170mm,257mm},
 left=20mm,
 top=20mm,
 }
\graphicspath{ {./images/} }

\begin{document}

\title{Unit 8 Task 5}
\author{Chris Hadden}
\date{}
\maketitle

\section{(P5) Follow the project plan and conduct a phase review of the identified project}
\subsection{Time}
All the activity will have been allocated a start and finish date before hand so that was easier to know what and when the task should be done. IF the task toke long than expect, the project manage will had to consider:
\begin{enumerate}
	\item Is the task on a critical pathway? Is failing to meet the deadlines likely to result in delaying completion of the project
	\item How can it be brought back on track?
	\item Is there a case to spend additional resources to bring the activity to a timely completion or at least to reduce the delay>
\end{enumerate}

\subsection{Quality}
 The project manager has to monitor the quality of all the processes being carried out by his team:
 \begin{enumerate}
 	\item Are the processes in line with the procedures laid down?
	\item Are they ensuring that the deliverables are being created in a timely manner to the required standard?
\end{enumerate}

Keeping to the plan that we designed in previous tasks makes it easy to keep to the quality under control. When we do make changes of our own we try to stick a strict quality plan so that quality is made and meet so that the client is happy with the project. Making sure the client is happy and pleased with the project is the purpose of a good quality product 

Quality management

A good quality management plan identifies and manages these quality standards.

Quality audits throughout the project review the quality standards and the assumptions made at the start of the plan to ensure that the standards are being maintained, including under budget and to the time agreed


\subsection{Change}
Changes made in our games project is going hard since the game project as these projects are big in scale and complex.

The project manager needs to consider any proposed changes to the project, the reasons for the requests and the effect that change will have on the timeliness and costs contained in the plan.

The project plan should be updated and the changes approved with the project sponsor.

\subsection{Risks}
The risk in this game development have all been thought of ahead of time to make sure that all risks are minimised. The current risks are potentially budget over run, development time over run and player feedback.
The risks identified in the plan need to be monitored. As the project develops, some risks may no longer be relevant and this is recorded on the plan. 
Other risks may arise, such as changes or additions to the deliverables being requested by the sponsor or the departure of a team member with specialist knowledge

\subsection{Issues}
Issues are not the same as risk. Here is why risk are reasons why risk and issues are different
\begin{enumerate}
	\item Risk can be anticipated
	\item Issues arise unexpectedly
\end{enumerate}

In this project as the project manager I must ensure that:
\begin{itemize}
	\item Issues are dealt with as a priority
	\item Issues are contained
	\item If an issue starts to take over the project then I let the client know as soon as possible
	\item Any changes to the plan that happens from issues are recorded and told to client
\end{itemize}

\subsection{Communications}
All communications between the client and us need to be clear and concise.
It is important that all communications are recorded so that the can be later used as a reference
We have agreed to have regular meetings with the client and to record those as minutes.
If any changes occur due to these communications then the change must be recorded.

\subsection{Acceptance}

Phase review for acceptance
The client must sign off all stakeholder documents and requirements.
Once this happens then we will have acceptance.

A phase review
		\subsection{Is the project on schedule}
		\subsection{Is the project within budget}
		\subsection{Have the deliverables been produced and approved}
		\subsection{Have the risks been controlled and mitigated}
		\subsection{Have issues been resolved}
		\subsection{Is the project on track}
		\subsection{Should the project contriun}


\section{(D3) Prepare a project closure report based on the execution phase review of the identified projectK}

\subsection{actions identified in the phase review that had to be addressed so that the project could be closed}

The actions that need to successfully close the project are
\begin{enumerate}
	\item Get all the adverts out 
	\item Make sure the client is happy everything
	\item Make sure all the deliverables were met
	\item Ensure the game is running with minimal bugs
	\item Ensure that game server is able to handle the player load
\end{enumerate}

\subsection{project summary}
The project summary is that we were asked to create a game for our client that would demonstrate their core ethos. In this case this was a fast food business. In working with the client we made a ninja 2d platform game that used elements of the foods that the client makes and sells.
We were asked to promote the healthy foods the client makes, and we did this. 
All the standard documents, risk listing and recording were all done and the client was kept in contact about any changes that were needed.
A server was setup to run the games, and play testing was done to make sure the players liked the game.


\subsection{reason for project closure(has the project reached its planned end date or is it being closed prematurely for some reason)}
The project was closed because we had done everything our client wanted and they closed off the project on there end after a successful run. The project closed when the game no longer served it's purpose of selling and promoting our clients food/


\subsection{assessment of project performance(how well did the project perform against the business case)}
Assessment of the project performance went well since our clients since a 25\% increase in their sales after releasing our game out to the world. We went with the requirements that we had set, that where "encourage end-users to purchase our client's food products" and to have each level themed after a specific food item form our client's selection. 

\textbf{The performance compared to the planned time } 
When comparing our project against the outline cost and planned time we have successfully hit our goals of being on time and with in budget. Keeping in budget was hard but with the correct managers it was easy since with having a gantt chart showed us what task is on time and which are late. Since we where able to see this it made it easy for us to adjust other task to fit in with the time window and make sure task didn't exceed there limit. The gantt chart also helps with the budget as well since time cost money and the longer time a task takes the more money we spent.

For the future there where areas we could have saved money. These areas are
\begin{enumerate}
	\item achievement of the project objectives, outputs and activities
	\item performance against planned time, resources and cost
	\item performance against planned savings and benefits
\end{enumerate}

\subsection{lessons learned}
The lessons learned from this experience are that working directly with our client saves everyone time and money involved. Keeping on track is hard but rewarding task to do. It might cause it bit more stress  but it means that we all save our time and money which I feel is more important to save than going over time and cost.

\subsection{celebrating success}
Celebrating success is quite important since the team needs an end to the project so that they don't feel over worked and under stress all the time, so celebrating is all so a big part in the project.


\subsection{next steps}
\subsubsection{outstanding activities}
There is no current outstanding activities since they where all done before submitting our game to the client/

\subsubsection{remaining risks and issues}
There might be a possibility that our game may cause some outrage with parents since we are promoting foods to young adults/children. Our client is prepared to deal with the mad parents. Other issues that might appear after releasing the game, these could be
\begin{enumerate}
	\item There might be bugs that we couldn't find during the end user testing. 
	\item Players may get bored
	\item The servers can not deal with the player load
\end{enumerate}

\subsubsection{on-going dependencies}
On-going dependencies might be that we might have to look after the servers. The server will need to have regular security updates and we will be tasked with that.
Some of the web technologies may become outdated in browsers, like Adobe flash did. In that case we may be hired again by the client to fix the game.
Bugs may be exposed only after the game is release. We will need to provide support for a specific amount of time according to the contract.


\subsubsection{costs of on-going support}
The costs of support are
\begin{enumerate}
	\item Developer bug fix time
	\item Support teams updating the game server
	\item Cost of running a server, cloud hosting or electricity
\end{enumerate}	

\subsubsection{stakeholder communications}
Regular meetings with clients is necessary to ensure that the client is happy with development as it continues. These meetings need to be recorded and agreed with the client.
We also have to communicate the initial plans and also any issues that come up, especially risks. This is very important if the issue affects cost or delivery time.

\subsubsection{handover of asseets  and contracts}
The handing over of assets will depend on what the client has payed for.
If they payed for all the source code and art work then they will get everything we developed, at that point they are free to do whatever they want with them. If they did not pay for thatn then  we keep them in house and if the client wants to use them again or create a new game with them then they will have to ask us to do the work.
We will keep the player server ourselves as we host other games there, the client did not want to setup and maintain their own server.



Project closure report will be

Project closure execution 



\end{document}
