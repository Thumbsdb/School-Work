\documentclass{article}
\usepackage{graphicx} % Required for inserting images
\usepackage[colorlinks=true, allcolors=blue]{hyperref}
\usepackage[square,numbers]{natbib}
\usepackage{titlesec}
\usepackage{geometry}
\geometry{
a4paper,
total={170mm,257mm},
left=20mm,
top=20mm,
}

\title{Unit 23. (M1) Describe the drawbacks of an identified cognitive computing application}
\author{Chris Hadden}
\date{}
\bibliographystyle{abbrvnat}

\begin{document}

\maketitle

\section{Cost}\label{cost}
There are several things that could influence the costs in cognitive computing:

\subsection{Advantages}
The best is advantage of using cognitive computing in healthcare is the same as in most other industries theres less processes and less staff needed. Cognitive AI is good at examining vast amounts of data and spotting patterns, similar to the tasks performed by a GP or nurse. Presenting AI with patient data allows it to identify health issues even before patients are aware of them. Preventive medicine like does, "prevent 40\% of all cancers" \cite{prevention}.

Chatbots like WatsonX can also replace receptionists and help by helping to diagnose patients. This could be seen as a cost-saving measure, although people whos jobs are under threat would consider it a disadvantage.

\subsection{Disadvantages}
Looking at the cost disadvantages of adopting cognitive computing technology in healthcare, it may initially seem that they might be minimal due to the ease of understanding and implementing this technology. However running cognitive computing is expensive you have to buy a lot of computers to run them.

\begin{itemize}
\item Training Costs Cognitive computing systems typically need a lot of data to learn and make accurate predictions. 

\item Hardware and Infrastructure Hospitals and GP surgeries typically do not have access to large computing infrastructure

\item Scalability Scaling cognitive can require more resources. If this solution is used only in a GP surgery, this maybe ok. For something like the NHS, it could be very costly.
\end{itemize}

Cognitive computing technologies can offer advantages better data analysis, and decision-making capabilities. The costs need careful planning and analysis are required to understand the total cost and make sure benefits justify the investments.

\section{Limited Analysis of Risk}
Analyzing risk is important when dealing with patients, as any mistakes can be deadly.

It's hard to know the risk with cognitive computer as its very unpredicatable and hard to tell whats happening. A few key points to consider 

\begin{itemize}
	\item Complexity and Predictability Cognitive computing systems can be are difficult to predict or explain, even to the developers. This unpredictability makes it hard to work out risks.
	\item Data Dependency Cognitive computing systems rely on their training data they are trained on. These can have a bias
	\item Human Oversight Human oversight remains crucial. AI should not be allowed to prescribe medicines.
\end{itemize}

We should consider these approaches:
\begin{itemize}
	\item Thorough Testing and Validation
	\item Data Governance
	\item Transparency and Documentation
\end{itemize}

\section{Security}
\subsection{Advantages}
Cognitive AI can have unexpected advantages in security. Being able to process large amounts of data allows an AI to learn about patients. There are already many examples of this \cite{security} in other areas, and it is possible to see how this could extend into healthcare.

\subsection{Disadvantages}
The security of patients is very important. Cognitive computing, while powerful has security issues:
\begin{itemize}
	\item Data Privacy  Any leaks will be devastating for patients. Current chatbots have already leaked data \cite{wired}.

	\item Bias .Security flaws can also come from the biases present in the training data or the algorithms themselves which it can lead to incorrect or harmful decisions. This could happen with healthcare training data if only data from white people was used.
\end{itemize}

Fixing these security problems will take a lot of work.

\section{Adoption}
As seen from the Cost \ref{cost}, where there was mention of replacing receptionists with a chatbot, those people concerned about losing their jobs will not want to adopt any AI. One way to stop this is to retrain anyone whose job is at risk. Any adoption of cognitive AI will need human moderation, which could be an area for redeployment. 
AI is supposed to work with people and does not replace them so an AI could talk to patients and pass them to a receptionist who decides what to do with them. This would free up receptionists to help do other stuff.

Adoption by patients could be harder to fix because of
\begin{itemize}
	\item Fear
	\item Unfamiliarity
	\item No feelings from AI
\end{itemize}
For feelings the research paper "Comparing Physician and Artificial Intelligence Chatbot Responses to Patient Questions Posted to a Public Social Media Forum" \cite{bedside} said "a chatbot generated quality and empathetic responses to patient questions posed in an online forum.".

\section{Change Management}
People can not help to resist change, which applies to both patients and healthcare staff. Cognitive AI should be implemented to work with people rather than replace them.

Implementing cognitive computing within an organization often needs a changes in management approaches. While these changes can have efficiency and decision-making, they also come with drawbacks like these

\begin{itemize}
\item Fear of losing job
\item Training and Skill Gaps
\item Healthcare has many different and incompatible systems.
\item Cognitive computing often has a culture towards more data-driven decision-making. This can be difficult.
\end{itemize}

\section{Lengthy Development Cycles}
AI by its nature has a long lead time before new updates can be released. This means that there can be a significant investment before a product is delivered, and fixes can take a long time to be developed.

\subsection{Advantages}
The healthcare sector is a huge and that is used to long lead times and expensive projects. Cognitive AI development may take as long or is already used to or could even be quicker.

\subsection{Disadvantages}
Cognitive AI is a big change in how people work. Trying to make this happen across many healthcare may be hard or even impossible. Acquiring the correct data for training from many different place may also increase the time an AI takes to develop.

Typically, in most software development practices, an iterative approach is preferred where the customer gets to try out early versions of code. AI does not work this way so it is possible that a lot of time and effort could go into developing something they didn't want.

\break
\bibliography{bibliography}

\end{document}
